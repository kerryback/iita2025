\documentclass[10pt]{beamer}

\usetheme[progressbar=foot]{metropolis}
%\usepackage{appendixnumberbeamer}

%\usepackage{booktabs}
\usepackage[scale=2]{ccicons}

\usepackage{pgfplots}
\usepgfplotslibrary{dateplot}
\pgfplotsset{compat=1.17} 

\usepackage{xspace}
\usepackage{xcolor}


\setbeamertemplate{frame footer}{MGMT 675: Generative AI for Finance}

\title{MGMT 675: Generative AI for Finance}
\subtitle{FMA 2025}

\date{October 24, 2025}
\author{Kerry Back\\ 
Jones Graduate School of Business\\
Rice University}


\begin{document}

\maketitle

\begin{frame}{Main Messages}
    \begin{enumerate}
    \item We should treat AI as a colleague, collaborator, and advisor, as well as an assistant. 
\item Large language models (LLMs) cannot be relied upon to do arithmetic, so coding (Python) is essential.  
\item AI + coding (vibe coding) can perform many financial analyses as well as or better than Excel.
\item For critical operations, we should save tested code as an app.
\item Chatbots can be customized through the system prompt, retrieval augmented generation (RAG) or fine tuning.
\item To use a chatbot for critical operations, we should create an app and configure it as a chatbot tool, creating an AI agent.
    \end{enumerate}
\end{frame}

\begin{frame}
To be continued...
\end{frame}
\end{document}