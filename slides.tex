\documentclass{beamer}

\usetheme[progressbar=foot]{metropolis}
%\usepackage{appendixnumberbeamer}

%\usepackage{booktabs}
\usepackage[scale=2]{ccicons}

\usepackage{pgfplots}
\usepgfplotslibrary{dateplot}
\pgfplotsset{compat=1.17} 

\usepackage{xspace}
\usepackage{xcolor}
\usepackage{graphicx}
\usepackage{hyperref}
\hypersetup{
    colorlinks=true,
    linkcolor=blue,
    urlcolor=cyan,
    citecolor=green
}


\setbeamertemplate{frame footer}{MGMT 675: Generative AI for Finance}

\title{MGMT 675: Generative AI for Finance}
\subtitle{FMA 2025}

\date{October 24, 2025}
\author{Kerry Back\\ 
Jones Graduate School of Business\\
Rice University}


\begin{document}

\maketitle


\begin{frame}{Overview}
    \begin{itemize}
    \item Half-semester MBA course at end of 1st year
    \item Course comes after: 
    \begin{itemize}
    \item Core finance (semester) 
    \item Excel-based Applied Finance (half-semester)
    \end{itemize}
    \item Finance topics mostly repeated from prior courses
    \item Ideas apply to courses of different lengths, at different points in the curriculum, and for different student groups
    \end{itemize}

\end{frame}


\begin{frame}{CNBC on JPMorgan, 9/30/2025}

Derek Waldron, , Chief Analytics Officer:
\begin{itemize}
\item (What we're working towards is that) every employee will have their own personalized AI assistant; every process is powered by AI agents, and every client experience has an AI concierge.
\item You'll still have people at the top who are managing and have relationships with clients, but many, many of the processes underneath are now being done by AI systems.
\end{itemize}


Workers would shift from being creators of reports ... or "makers" ... to "checkers" or managers of AI agents doing that work.

\end{frame}

\begin{frame}{Yeyati, Brookings, 2025}

\begin{itemize}
\item As AI models begin to handle underwriting, compliance, and asset allocation, the traditional architecture of financial work is undergoing a fundamental shift.
\item As job descriptions evolve, so does the definition of financial talent. Excel is no longer a differentiator. Python is fast becoming the new Excel. 
\item But technical skills alone will not cut it. The most in demand profiles today are those that speak both AI and finance.
\end{itemize}
\end{frame}

\begin{frame}{Learning Objectives}
\begin{enumerate}
\item How to work with AI to do financial analysis
\item How custom chatbots and AI agents work 
\item How to work with AI to build custom chatbots and AI agents for financial analysis
\end{enumerate}
\end{frame}

\begin{frame}{Website: \href{https://genai4finance.kerryback.com}{genai4finance.kerryback.com}}
\begin{itemize}
\item Course description aimed at instructors
\item Blog: short posts about teaching various topics on AI and finance 
\item Course materials: 2025 and 2026 (partial)
\item Slides (these and upcoming talk)
\item Python materials (pre-course workshop or individual study)
\end{itemize}
\end{frame} 

\begin{frame}{Plan}
    \begin{itemize}
    \item Skim website materials
    \item Demo representing in-class exercise
    \end{itemize}
\end{frame}

\begin{frame}{Some AI Tools}
    \begin{itemize}
    \item Claude: Claude Desktop, Claude.ai, and Claude Code 
    \item ChatGPT (and custom GPTs)
    \item Google Colab and Gemini
    \item Julius.ai
    \end{itemize}
\end{frame}

\begin{frame}{Julius Demo Prompt 1}
   \begin{quote}
   Use the latest version of yfinance to get closing prices at a monthly frequency from Yahoo Finance for SPY, IEF, and GLD since 1970. Compute returns as percent changes and filter to the longest history for which returns for all three ETFs are available. Compute the historical mean and covariance matrix. Compute the tangency portfolio assuming the monthly risk-free rate is 0.04/12.

   \end{quote}
\end{frame}

\begin{frame}{Julius Demo Prompt 2}
   \begin{quote}
   Create a professionally formatted Word doc containing a plot containing (i) the mean-variance frontier of risky assets, (ii) the risk-free rate, (iii) the tangency portfolio, and (iv) the capital allocation line extending through the tangency portfolio.  Include in the Word doc the historical means, standard deviations, and correlations of the monthly SPY, IEF, and GLD returns computed before, an explanation of the method used to compute the tangency portfolio and your interpretation of why the tangency portfolio is what it is.

   \end{quote}
\end{frame}


\end{document}


